\documentclass{tufte-handout}

%\geometry{showframe}
%\geometry{showframe}% for debugging purposes -- displays the margins

\usepackage{amsmath}
\usepackage{cleveref}
\crefname{Proposition}{Proposition}{Propositions}
\Crefname{Proposition}{Proposition}{Propositions}
\crefname{Theorem}{Theorem}{Theorems}
\Crefname{Theorem}{Theorem}{Theorems}
\crefname{Definition}{Definition}{Definitions}
\Crefname{Definition}{Definition}{Definitions}
\crefname{Corollary}{Corollary}{Corollaries}
\Crefname{Corollary}{Corollary}{Corollaries}
\crefname{Lemma}{Lemma}{Lemmas}
\Crefname{Lemma}{Lemma}{Lemmas}
\crefname{Example}{Example}{Examples}
\Crefname{Example}{Example}{Examples}
\usepackage{amsthm}

% Set up the images/graphics package
\usepackage{graphicx}
\setkeys{Gin}{width=\linewidth,totalheight=\textheight,keepaspectratio}
\graphicspath{{graphics/}}

% The following package makes prettier tables.  We're all about the bling!
\usepackage{booktabs}

% The units package provides nice, non-stacked fractions and better spacing
% for units.
\usepackage{units}

% The fancyvrb package lets us customize the formatting of verbatim
% environments.  We use a slightly smaller font.
\usepackage{fancyvrb}
\fvset{fontsize=\normalsize}

% Small sections of multiple columns
\usepackage{multicol}

% Provides paragraphs of dummy text
\usepackage{lipsum}

% Defines colors
\usepackage{xcolor}
\definecolor{blue}{cmyk}{0.63, 0.37, 0, 0.57}     
\definecolor{ltblue}{RGB}{78,150,179}

% These commands are used to pretty-print LaTeX commands
\newcommand{\doccmd}[1]{\texttt{\textbackslash#1}}% command name -- adds backslash automatically
\newcommand{\docopt}[1]{\ensuremath{\langle}\textrm{\textit{#1}}\ensuremath{\rangle}}% optional command argument
\newcommand{\docarg}[1]{\textrm{\textit{#1}}}% (required) command argument
\newenvironment{docspec}{\begin{quote}\noindent}{\end{quote}}% command specification environment
\newcommand{\docenv}[1]{\textsf{#1}}% environment name
\newcommand{\docpkg}[1]{\texttt{#1}}% package name
\newcommand{\doccls}[1]{\texttt{#1}}% document class name
\newcommand{\docclsopt}[1]{\texttt{#1}}% document class option name

% Package for title style
\usepackage{sectsty}
\usepackage[utf8]{inputenc}

% Sets section number style
\setcounter{secnumdepth}{3} % uncomment this, if desired
\renewcommand\thesection{\color{white}\arabic{section}}

% Sets title style
\makeatletter
  \renewcommand{\paragraph}{\@startsection{paragraph}%
    {4}{\z@}{-1ex \@plus -1ex \@minus -.3ex}%
    {0.5ex \@plus .2ex}{\normalfont\normalsize\bfseries}}
\makeatother

\makeatletter
  \renewcommand{\subsection}{\@startsection{subsection}%
    {3}{-1.8em}{-3ex \@plus -1ex \@minus -.2ex}%
    {1.5ex \@plus .2ex}
    {\hspace*{-5.5em}\fcolorbox{ltblue}{ltblue}{\parbox[c][1.0ex][b]{4em}{\phantom{space}}}
    \normalfont\large\itshape\color{ltblue}}}
\makeatother

\makeatletter
  \renewcommand{\section}{\@startsection{section}%
    {3}{-1.01em}{-3ex \@plus -1ex \@minus -.2ex}%
    {1.5ex \@plus .2ex}
    {\hspace*{-5.5em}\fcolorbox{blue}{blue}{\parbox[c][1.0ex][b]{4em}{\phantom{space}}}
    \normalfont\Large\itshape\color{blue}}}
\makeatother

% Sets theorem style
\usepackage{thmtools}
\usepackage{transparent}
\definecolor{theb}{rgb}{0.67, 0.80, 0.91}

\declaretheorem[shaded={bgcolor=Lavender,
    textwidth=30em}]{Definition} % Colorbox-styled Theorem 
\declaretheorem[shaded={bgcolor=Lavender,
    textwidth=30em}]{Theorem} % Colorbox-styled Theorem 
\declaretheorem[shaded={bgcolor=Thistle,
    textwidth=30em}]{Corollary} % Colorbox-styled Theorem
\declaretheorem[shaded={bgcolor=PeachPuff,
    textwidth=30em}]{Proposition} % Colorbox-styled Proposition
\declaretheorem[shaded={bgcolor=Thistle,
    textwidth=30em}]{Lemma} % Colorbox-styled Lemma
\declaretheorem[shaded={rulecolor=Lavender,
    rulewidth=2pt, bgcolor={rgb}{1,1,1}}]{Example-1} % Colorbounded-styled Theorem
\declaretheorem[thmbox=L]{boxtheorem L} % Theorem box L-size
\declaretheorem[thmbox=M]{Example} % Theorem box M-size
\declaretheorem[thmbox=S]{Formula} % Theorem box S-size
\declaretheorem[thmbox=S]{Statement} % Theorem box S-size

% set up pdf bookmark depth
\hypersetup{bookmarksdepth=3}

% ------------------------------------------------------------
\title{MATH240: Discrete Structures}
\author[Matthew He]{Matthew He}
  % if the \date{} command is left out, the current date will be used

% Beginning of the document
\begin{document}

\maketitle% this prints the handout title, author, and date

% abstract
\begin{abstract}
\noindent Abstract
\end{abstract}

% main text
\section{Predicate Logic}
    \begin{Definition}[Predicate]
    A \it predicate \rm is a statement containing some number
    of variable coming from a universe \textbf{u}. aaa
    \end{Definition}

\subsection{Tutorial}
\begin{enumerate}
    \item $ p \Rightarrow p $: tautology, since 
    $ \equiv \neg p \vee p = 1$ 
    \item $(p \Rightarrow q) \Rightarrow p$: contingency,
    true if $ p=q=1 $, false if $ p=0 $. 
    \marginnote{When $ p = 0 $, $ p \Rightarrow q  $is always true, and 
    a $ture \Rightarrow false (p)$ is always false.}
    
    \item $ (\neg (p \wedge \neg q)) \vee p \equiv 1 $: tautology. 
    \item $ (p \Leftrightarrow q) \wedge (p \Leftrightarrow \neg q )$: contradictory. 
    \item to be filled
\end{enumerate}

\begin{Example}
    Prove that a logical formula is satisfiable iff. its 
    negation is falsifiable.
\end{Example}
\textit{Proof.} A logical formula is satisfiable iff there is an assignment of 
all the variables which makes the formula true. By definition of the negation, this assignment makes the negation of our formula false, 
which means the negation is falsifiable. Proof of the converse is analogous.

\paragraph{Prove or disprove:}
\begin{enumerate}
  \item $ \forall n \in \mathbb{N }, \exists m \in \mathbb{N }, n + m = 0 $ \\
  False. Let's proof its negation: $ \exists n \in \mathbb{N }, \forall m \in \mathbb{N },
  n + m \neq 0$, it's true, e.g., we can let $ n = 2 $.
  \item $ \forall n \in \mathbb{N }, \exists m \in \mathbb{Z }, n + m = 0 $\\
  Ture, Let $ n \in \mathbb{N } $, then choose $ m = -n \in \mathbb{Z }  $, and we get $ m+n=0 $. 
  \item $ \forall n \in \mathbb{N }, \exists k \in \mathbb{N }, (k \geq m) \Rightarrow (k \geq 5n) $: \\
  The statment is equivalent to $ (k < m) \vee (k \geq 5n)$. \\
  For any $ n \in \mathbb{N }$, choose $m = 5n $, then for all $ k \in \mathbb{N } $, we have that statement is true.
\end{enumerate}

\subsection{Proofing statement of the form $ p \Rightarrow q $}
    We assume p is true, prove q, which is showing the case that p is true 
    be q is false can't happen.
    \begin{Example}
        \textit{prop.} If n is an odd integer, then $n^{2} $ is an odd integer.
    \end{Example}
    Write the proposition in predicate formula:
    \[u = \mathbb{Z }, \phantom{x} \forall n: ((\exists k : n = 2k+1) \Rightarrow (\exists n^{2} = 2l+1))\]
    \textit{proof.} Let n be ana integer. Assume that n is odd, that is, there exist k such that $ n = 2k+1 $.\\
    Then, $ n^{2} = (2k+1)^{2} = 2(2k^{2} + 2k) + 1 $. Let $ l = 2k^{2} + 2k $,
    then $ n^{2} = 2l + 1 $, thus, it's odd.

\subsection{To disprove a statement: prove its negation is true}



\begin{Example}
    disprove $ \exists x \forall y: x + y \neq 0  $. \\
    prove $ \forall x \exists y : x+ y = 0 \equiv \neg (\exists x \forall y: x + y \neq 0) $
\end{Example}
\textit{proof.} Let $ x \in \mathbb{R } $ be given. Pick $ y = -x  $, then 
$ x + y = 0 $. Q.E.D

\marginnote{Since $ p \wedge \neg p \equiv 0 $}

\subsection{Converse and Contrapositive}

\begin{Definition}\phantom{x }\\
    \begin{enumerate}
        \item the \textbf{converse} of $ p \Rightarrow q  $ is $  q \Rightarrow p  $\\
        NB. $ p \Rightarrow q  \not\equiv  q \Rightarrow p  $
        \item the \textbf{contrapositive} of $ p \Rightarrow q  $ is$\neg q \Rightarrow \neg p  $ \\
        $ p \Rightarrow q  \equiv \neg p \vee q \equiv \neg q \Rightarrow \neg p$
    \end{enumerate}
\end{Definition}

\subsection{Proofs by contradiction}
You assume something is true, and get something nonsense.\\
\[\neg p \Rightarrow 0 \equiv \neg (\neg p) \vee 0 \equiv p\]

\begin{Example}
    \textit{prop.} there is no least positive rational number.\\
    \[u = \mathbb{Q }: \neg (\exists x : x>0 \wedge (\forall y: y > 0 \Rightarrow x \leq y))\]
\end{Example}
\textit{proof.} Suppose, for a contradiction that the proposition is false, that is,
there exist $  x \in \mathbb{ Q } $ such that $ x>0 $ and for all $ y \in \mathbb{Q } $ with $ y>0, x \leq y  $.\\
Let $ y = \frac{x }{2}$, we have $ \frac{x }{2} > 0 $ since $ x>0 $. Then $ x \leq y  $, so $ x \leq \frac{x }{2} $.\\
Divide through by x ( because x > 0) to get $ 1 \leq \frac{1}{2}$.\\
the contradiction completes the proof. Q.E.D

\subsection{Case Analysis}
\begin{Example}
    \textit{prop.} There exists irrational numbers a,b such that a is rational.
\end{Example}

If 

\section{Functions}
\begin{Definition}[Surjective and injective]
    A function is surjective if 
    \[\forall b \in B, \exists a \in A, f(a) = b\]
    A function is injective if 
    \[\forall a_1, a_2 \in A, a_1 \neq a_2 \rightarrow f(a_1) \neq f(a_2)\]
    or $ a_1 = a_2 \rightarrow f(a_1) = f(a_2) $.
\end{Definition}

\section{Modular Arithmetic}

\subsection{Division}
\begin{Proposition}
    Let a and b be nonzero integers. The set
    \[X = \left\{s'a+t'b:s',t'\in \mathbb{Z}\right\}\]
    is exactly the set of multiples of $ d = \gcd(a,b) $.
\end{Proposition}
\textit{Proof.} By Bezout's identity, there exist integers $ s,t $ such that \( d = sa+tb \).
First let \( n \in \mathbb{Z} \) be a multiple of \( d \), so \( n = kd \) for some \( k \in \mathbb{Z} \).
Then we have \[n=kd=d(sa+tb)=(ds)a+(dt)b,\]
which means that \( n\in \mathbf{X} \), since \( ds,dt \in \mathbb{Z} \).

Conversely, suppose that \( n \in \mathbf{X} \). Then \( n = s'a+t'b \) for some \( s',t' \in \mathbb{Z} \).
Then since \( d \) divides \( a \) and \( b \), we can write \( a=ld \) and \( b=md \) for some integers \( l,m \in \mathbb{Z}\).
So we have \[n=s'a+t'b=s'ld+t'md=(s'l+t'm)d,\]
which shows that \( d|n \), since \( s'l+t'm \in \mathbb{Z} \). \qedsymbol

\section{Graph Theory}
\subsection{Definitions}
A graph is a pair $ G = (V,E) $, where V is a nonempty set and \[E \subseteq \left\{ {u,v}: u,v \in V, u \neq v\right\}.\]

\marginnote{The graph \( G \) is said to be \textit{finite} if both \( V \) are finite sets.
A simple graph is a graph that does not have more than one edge between any two vertices and no edge
starts and ends at the same vertex.}

\textbf{Degrees and k-regularity}
The \textit{neighbors} of a vertex \( v \) are all \( u \in V \) such that \( uv \in E \). The 
\textit{degree} of a vertex \( v \) is the number of neighbors of \( v \), denoted by \( deg(v) \).
A graph is said to be \( k-regular \) for some \( k\in \mathbb{N} \) if every \( v \in V \) has degree \( k \).

The following theorem relates vertex degree to the number of edges.
\begin{Theorem}
    Let G = (V,E) be a finite graph. Then \[\sum_{v \in V} \deg(v) = 2 \left|E\right|.\]
\end{Theorem}

\begin{Corollary}[Handshaking Lemma]
    In every finite simple graph, the number of vertices having odd degree is even.
\end{Corollary}

From Theorem above, we can derive a corollary that counts the number of edges in k-regular graphs.

\begin{Corollary}
    Let G = (V,E) be k-regular. Then \[\left|E\right| = \frac{k\left|V\right|}{2}.\]
\end{Corollary}

\textbf{Walks, paths, and cycles.} A \textit{walk} in a graph \( G = (V,E) \)
is a sequence of vertices \( \sigma = (v_0, v_1, \ldots, v_k) \) such that \( v_i v_{i+1} \in E \) for all \( i \) with \( 0 \leq i < k \). 
The \textit{endpoints} of the walk are \( v_0 \) and \( v_k \), and the \textit{lenght} of the walk \( \sigma \) is \( \left|\sigma\right| \)
The walk is said to be \textit{closed} if \( v_0 = v_k \) and \textit{open} otherwise. 

A walk is a \textit{path} if no vertices are repeated. 
\begin{Theorem}
    Let G = (V,E) be a graph. If \( u \) and \( v \) are 
    vertices such that there exists a walk from \( u \) to \( v \), then there exists a path from \( u \) to \( v \).
\end{Theorem}

\marginnote{There we proform a minimality argument. Note that this theorem is about existence of path, a walk is not necessary a path.}

\textit{Proof.} Let \( \sigma = (v_0, v_1, \ldots, v_n) \) be a walk from \( u \) to \( v \) of shortest length.
We claim that \( \sigma \) is a path. Indeed, suppose for a contradiction that it is not a path; then there is some repeated vertex,
so there exist \( i,j \in {0,1,\ldots,n}\) such that \(i < j \) and \( v_i = v_j \).  Hence
there is no need to visit any of the vertices between \( v_i \) and \( v_j \) in the walk,
since \( v_i = v_j\) is connected to \( v_{j+1} \). Concretely, consider the walk 
\[\sigma'=(v_0, v_1, \ldots, v_i, v_{j+1}, \ldots, v_n).\]
Note that \( \left|\sigma'\right| = \left|\sigma\right| - (j-i) \), and \( j-i>0 \), so \( \sigma' \) is a
shorter walk from \( u \) to \( v \). But this contradicts our choice of \( \sigma \) as a walk of shortest length.
We conclude that \( \sigma \) is a path.
\qedsymbol

A \textit{cycle} is a walk of length at least 3 and no vertices repeated except for \( v_0 = v_k \).

\begin{Proposition}
    \label{prop:odd_cycle}
    Let G = (V,E). If G contains a closed walk of odd length, then G contains a cycle of odd length.
\end{Proposition}

\marginnote{The idea is similar to the proof of the previous theorem.}

\textit{Proof.} Let \( \sigma = (v_0, v_1, \ldots, v_n) \) be an odd-length closed walk
in G, and choose this walk to have minimal odd length (i.e, any shorter closed walk has even length).
We shall prove that \( \sigma \) is a cycle.

For a contradiction, suppose that \( \sigma \) is not a cycle, so that there exist indices \( i,j \) with \( 0 \leq i < j \leq n \)
such that \( v_i = v_j \). Consider the two closed walks
\[\sigma_1 = (v_0, v_1, \ldots, v_i, v_{j+1},\ldots , v_{n})\] and
\[\sigma_2 = (v_i, v_{j+1}, \ldots, v_j).\]
Both are shorter than \( \sigma \), so by the minimality of \( \sigma \), they must have even length.
But this implies that \( \left|\sigma\right| = \left|\sigma_1\right| + \left|\sigma_2\right| \) is even.
This leads to a contradiction.

We conclude that \( \sigma \) is a cycle. \qedsymbol

\textbf{Connectedness.} We say a graph \( G = (V,E) \) is \textit{connected} if for all \( u,v \in V \),
there exists a walk from \( u \) to \( v \).
\marginnote{ This is a disconnected graph:
\includegraphics*[width=\linewidth]{graphics/Disconnected_Graph}
}

\begin{Proposition}
    For all \( n \geq 1 \), the graph \( K_n \) and \( Q_n \) are connected.
\end{Proposition}

\textit{Proof.} Any two vertices \( u \) and \( v \) in \( K_n \) are connected by an edge, so
we have a path \( (u,v) \) of length 1 between \( u \) and \( v \). This shows that \( K_n \)
is connected.

Now let \( u \) and \( v \) be any two vertices in \( Q_n \). Suppose there are m 
bits that differ between \( u \) and \( v \), Then we can flip them one by one to change \( u \) to \( v \).
This gives us a walk of length \( m \) between \( u \) and \( v \), since there is an
edge in \( Q_n \) between any two strings in \( Q_n \) that differ at exactly one bit. \qedsymbol

Here is an example:
\begin{Example}[Disconnected Graph and Modular Arithmetic]
    Let \( G = (V,E) \), where for \( i,j \in \mathbb{Z} \) with \( i<j \), we have \( ij \in E \)
    if and only if \( j-i \in {9,15} \). Then \( G \) is disconnected.
\end{Example}
\textit{Proof.}
Starting at \( n \in \mathbb{Z} \), we can reach any vertex \( m \) that is of the form \[m=n+15s+9t\]
for some \( s,t \in \mathbb{Z} \). By proposition in division section, the integers representable as
\( n+15s+9t \) are exactly the multiples of \( \gcd(15,9) = 3 \). So in fact, from \( n \) one can reach any integer \( m \)
of the form \( n+3k \) for some \( k \in \mathbb{Z} \). That is, one can reach any integer \( m \) with
\( m \equiv n \mod 3 \). Hence the three connected components of \( G \) are \( [0]_3, [1]_3,[2]_3\),
the equivalence classes of integers modulo 3. \qedsymbol

\subsection{Triangles and bipartite graphs}
A \textit{subgraph} of a graph \( G = (V,E) \) is a graph \( G' = (V',E') \) such that \( V' \subseteq V \) and \( E' \subseteq E \)
where for all \( e=uv \in E' \), we have \( u,v \in V' \).

\textbf{The extremal question} An extremal question asks for the extermal (maximum or minimum) number of objects
we can have, subject to some restrictions.

We now want to derive the maximum number of edges in a triangle-free graph on \( n \) vertices.

\begin{Theorem}[Cauchy-Schwarz Inequality] \hfill\\
    For all \( u_1, \ldots, u_n, v_1, \ldots, v_n \in \mathbb{R} \), we have\[\left(\sum_{i=1}^{n} u_i v_i\right)^2 \leq \left(\sum_{i=1}^{n} u_i^2\right)\left(\sum_{i=1}^{n} v_i^2\right).\]
\end{Theorem}

\begin{Theorem}[Mantel's theorem]
    Let \( G = (V,E) \) be a graph not containing a triangle as a subgraph. Then
    \[ \left|E\right| \leq \lfloor \frac{\left|V\right|^2}{4} \rfloor.\]
\end{Theorem}

\textit{Proof.} Consider the sum \[ \sum_{uv\in E}\left(\deg(u)+\deg(v)\right).\]
The term \( \deg(u) \) appears in the sum exactly once for every edge incident on u; that is, it
appears \( \deg(u) \) times. This is true for all \( u \in V \), so we conclude that
\[ \sum_{uv\in E}\left(\deg(u)+\deg(v)\right) = \sum_{u\in V}\deg(u)^2.\]
On the other hand, since G contains no triangle, for every pair of vertices \( u\) and \( v \), the set
of neighbors of \( u \) is disjoint from the set of neighbors of \( v \). So \( \deg(u) +\deg(v) \leq \left|V\right| \), and we have
\[ \sum_{u\in V}\deg(u)^2=\sum_{uv\in E}\left(\deg(u)+\deg(v)\right) \leq \left|E\right|\left|V\right|.\]
\marginnote{
Derivation using Cauchy-Schwarz inequality:
\begin{align}
    (2\left|E\right|)^2 & = \left(\sum_{u\in V}\left(\deg(u)\right)\right)^2 \\
    &= \left(\sum_{u\in V}\deg(u)\cdot 1\right)^2 \\
    & \leq \left(\sum_{u\in V}\deg(u)^2\right)\left(\sum_{u\in V}1^2\right) \\
    &= \left|V\right|\left(\sum_{u\in V}\deg(u)^2\right).
\end{align}}
By the Cauchy-Schwarz inequality, we have
\[(2\left|E\right|)^2\leq \left|V\right|\left(\sum_{u\in V}\deg(u)^2\right)\]
Hence \[4\left|E\right|^2\leq\left|V\right|\left(\sum_{u\in V}\deg(u)^2\right)\leq \left|V\right|^2\left|E\right|.\]
This implies that \( \left|E\right|\leq \frac{\left|V\right|^2}{4} \).
We can take the floor function on the R.H.S, since \( \left|E\right| \) must be an integer.\qedsymbol

So if a graph has \(\left|E\right| > \lfloor \left|V\right|^2\slash 4 \rfloor \), there must be a triangle subgraph in G.
\marginnote{This is the converse of the proposition we just proved. Pay attention here, because that means the theorem does not assert
any graph has edges less or equal to \(\lfloor \left|V\right|^2\slash 4\rfloor \) is triangle-free.}

How about the case a graph has exactly \(\left|V\right|^2\slash 4 \) edges? 
The Mantel's theorem does not assert that it must have triangles.
This brings us to the definition of a bipartite graph.

\textbf{Bipartite graphs} A graph \( G = (V,E) \) is \textit{bipartite} if there exists
a partition of \( V = A \cup B \) of the vertex set (\( A\cap B = \emptyset\)) called the \textit{bipartition}
such that each edge has one endpoint in \( A \) and the other in \( B \).
For example, hypercubes \( Q_n \) are bipartite.

\begin{Proposition}
    For all \( n \geq 1 \), the graph \( Q_n \) is bipartite.
\end{Proposition}
\textit{Proof.} Let \( Q_n = (V,E) \). Every elements \( s\in V \) corresponds to a binary string of length \( n \),
\( S = (s_1, s_2, \ldots, s_n) \) where each \( s_i \) is either 0 or 1. Define
\[V_0 = \left\{s\in V: s_1 + \dots + s_n \equiv 0 \pmod 2\right\}\]
and
\[V_1 = \left\{s\in V: s_1 + \dots + s_n \equiv 1 \pmod 2\right\}.\]
It's clear that \( V_0 \cup V_1 =V \) and \( V_0 \cap V_1 = \emptyset \), so this is a bipartition of the vertex set.
For every \( e = s_1s_2 \in E \), the strings \( s_1 \) and \( s_2 \) differ in exactly on bit, so if
\( s_1 \in V_0 \), then \( s_2 \in V_1 \) and vice versa. Hence \( Q_n \) is bipartite. \qedsymbol

The \textit{complete bipartite graph} \( K_{m,n} \) is a bipartite graph with bipartition \( V = V_m \cup V_n \), where
\marginnote{
\includegraphics*[width=\linewidth]{graphics/complete_bipartite}
}
\( \left|V_m\right| = m \) and \( \left|V_n\right| = n \), and E is the set \( {uv: u\in V_m, v\in V_n} \) of
all possible edges between the two sets.

When \( n \) is even, the graph \( K_{n/2,n/2} \) has n vertices and exactly \( n^2\slash 4 \) edges.
When \( n \) is odd, the graph \( K_{(n-1)/2,(n+1)/2} \) has n vertices and 
\[\frac{n+1}{2}\cdot \frac{n-1}{2} = \frac{n^2-1}{4} = \frac{n^2}{4}-\frac{1}{4} = \lfloor \frac{n^2}{4} \rfloor\]
\marginnote{\( K_{n/2,n/2} \) has the largest possible number of edge shuch that the Mantel's theorem does not apply. But adding
a single edge to \( K_{n/2,n/2} \) results in a graph that must contain a triangle by Mantel's theorem.}
edges. (The last equality here follows from the fact that any odd \( n \) is congruent to either 1 or 3 modulo 4,
which means its square is congruent to 1 modulo 4.)

\begin{Lemma}
    A graph \( G = (V,E) \) is bipartite if and only if its connected components are bipartite.
\end{Lemma}

\textit{Proof.} Let \(G_1 = (V_1,E_1), \ldots, G_k = (V_k,E_k) \) be the connected components of \( G \).

If each connected component is bipartite, then we can bipartition each \( V_i \) into \( A_i \cup B_i \).
Then \( A = \bigcup_{i=1}^{n} A_i \) and \( B = \bigcup_{i=1}^{n} B_i \) is a bipartition of \( V \).
And every edge in \( G \) has one endpoint in \( A \) and the other in \( B \), so \( G \) is bipartite.

Now suppose that there is some connected component \( G_i \) that is not bipartite. Now let
\(V = A\cup B \) be any partition of \( V \) into disjoint nonempty sets. Then \(A_i=A\cap V\) and
\(B_i = B\cap V\)  form a partition of \( V_i \) into disjoint nonempty sets. But since \( G_i \) is not bipartite,
there must be an edge \( uv \in E_i \) with both endpoints in \( A_i \) or both in \( B_i \).
This means that \( A \cup B \) is not a bipartition of \( V \), and since A and B were arbitrary, we conclude that
\( G \) is not bipartite either. \qedsymbol

As a matter of fact, we shall prove something much stronger than just the fact that \( K_{n/2,n/2} \)
does not contain any triangles. 
\begin{Theorem}
    A graph is bipartite if and only if it does not contain any cycles of odd length.
\end{Theorem}
\marginnote{Proof of this theorem requires the metric of distance imposed on a graph G. For all 
\( u,v \in V \), the distance \( dist(u,v) \) is the length of the shortest path from \( u \) to \( v \).
If no such path exists, then \( dist(u,v) = \infty \).}
\textit{Proof.} First, assume that G = (V,E) is bipartite with bipartition \( V = A \cup B \).
Let \(\sigma\) be a cycle in G. Each edge changes sides between A and B, so in order
for starting and ending vertices of this cycle to be the same, \(\left|\sigma\right|\)
must be even.

Now suppose that G does not contain any cycles of odd length. To show that G is bipartite,
it suffices to show that each of its connected components is bipartite by the lemma above.
So without loss of generality we may assume that G is connected.
That means \(dist(u,v) < \infty\) for all \(u,v \in V\).

Select on vertex \( h\in V   \) and set
\[V_0 = \{v\in V: dist(h,v) \equiv 0 \pmod2\}\]
and
\[V_1 = \{v\in V: dist(h,v) \equiv 1 \pmod2\}.\]
Let \( e = uv \in E \). Consider a closed walk \(\sigma\) that follows a shortest path from \( u \) to \( h \)
then a shortest path from \( h \) to \( v \), and finally the edge \( uv \). The length of \(\sigma\) is
\[ \left|\sigma\right| = dist(u,h) + dist(h,v) + 1 \]
But since G contains no cycles of odd length, \(\left|\sigma\right|\) must be even, by (the Contrapositive of) \Cref{prop:odd_cycle},
any closed walk in G must have even length. This means that \( dist(u,h),dist(h,v)\) are not the same modulo 2.
That is, either \(u \in V_0\) and \(v \in V_1\) or vice versa. \qedsymbol

\subsection{Trees}

\begin{Proposition}
    Let F be a forest with at least one edge (and hence at least two vertices). There are at least two leaves in F.
\end{Proposition}

\begin{Proposition}
    A graph G = (V,E) is a tree if and only if for all \( u,v \in V \), there
    exist a unique path from \( u \) to \( v \).
\end{Proposition}

\begin{Lemma}[Detour Lemma]
    Let G = (V,E) be a connected graph and let \(\sigma\) be a cycle in \( G \),
    If \( G' \) is the graph obtained by deleting on edge from \(\sigma\), then \( G' \) is still connected.
\end{Lemma}

\begin{Lemma}
    A graph G = (V,E) is a tree if and only if it is connected and \(\left|E\right| = \left|V\right|-1\).
\end{Lemma}

\begin{Corollary}
    Let F = (V,E) be a forest, then \(\left|E\right| \leq \left|V\right| - 1\).
\end{Corollary}


\subsection{Eulerian trails and circuits}

\begin{Theorem}
    Let G = (V,E) be a connected graph with a finite number of vertices and edges, where
    multiple edges between the same pair of vertices are allowed. Then G has an Eulerian circuit
    if and only if every vertex in G has even degree.
\end{Theorem}

\begin{Theorem}
    Let G = (V,E) be a connected graph (with multiple edges allowed). Then G contains an Eulerian trail
    that isn't an Eulerian circuit if and only if exactly two vertices of G have odd degree.
\end{Theorem}

\subsection{Planar graphs}
\begin{Proposition}
    If G is planar then any subgraph H of G is also planar.
\end{Proposition}

\begin{Theorem}[Euler's formula]
    Let G = (V,E) be a connected planar graph, let \(f\) be the number of faces in a planar embedding of G.
    Then \( \left|V\right| - \left|E\right| + f = 2 \).
\end{Theorem}

\begin{Theorem}[Jordan curve theorem]
    Every closed curve in the plane \(\mathbb{R}^2 \) that does not intersect itself divides the plane into two regions.
\end{Theorem}

\begin{Theorem}
    Let G = (V,E) be a connected planar graph with \( \left|V\right| \geq 5 \). The
    \[\left|E\right| \leq 3\left|V\right| - 6.\]
    Under the further assumption that G contains no triangles, we have a better bound
    \[\left|E\right| \leq 2\left|V\right| - 4.\]
\end{Theorem}

\begin{Corollary}
    The complete graph \( K_5 \) and the complete bipartite graph \( K_{3,3} \) are nonplanar.
\end{Corollary}

\begin{Theorem}[Wagner's theorem]
    A graph G is nonplanar if and only if either \(K_5\) or \(K_{3,3}\) is a minor of G.
\end{Theorem}

\begin{Theorem}[Kuratowski's theorem]
    A graph \(G\) is nonplanar if and only if it contains a subdivision of either \(K_5\) or \(K_{3,3}\) as a subgraph.
    
\end{Theorem}

% end of main text

\makeatletter
  \renewcommand{\section}{\@startsection{section}%
    {3}{0.8em}{-3ex \@plus -1ex \@minus -.2ex}%
    {1.5ex \@plus .2ex}
    {\hspace*{-5.5em}\fcolorbox{Periwinkle}{Periwinkle}{\parbox[c][1.0ex][b]{4em}{\phantom{space}}}
    \normalfont\Large\itshape\color{blue}}}
\makeatother

\bibliography{marginnotes}
\bibliographystyle{plainnat}

\end{document}
