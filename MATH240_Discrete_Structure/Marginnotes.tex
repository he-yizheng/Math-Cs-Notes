\documentclass{tufte-handout}

%\geometry{showframe}
%\geometry{showframe}% for debugging purposes -- displays the margins

\usepackage{amsmath}
\usepackage{amsthm}

% Set up the images/graphics package
\usepackage{graphicx}
\setkeys{Gin}{width=\linewidth,totalheight=\textheight,keepaspectratio}
\graphicspath{{graphics/}}

% The following package makes prettier tables.  We're all about the bling!
\usepackage{booktabs}

% The units package provides nice, non-stacked fractions and better spacing
% for units.
\usepackage{units}

% The fancyvrb package lets us customize the formatting of verbatim
% environments.  We use a slightly smaller font.
\usepackage{fancyvrb}
\fvset{fontsize=\normalsize}

% Small sections of multiple columns
\usepackage{multicol}

% Provides paragraphs of dummy text
\usepackage{lipsum}

% Defines colors
\usepackage{xcolor}
\definecolor{blue}{cmyk}{0.63, 0.37, 0, 0.57}     
\definecolor{ltblue}{RGB}{78,150,179}

% These commands are used to pretty-print LaTeX commands
\newcommand{\doccmd}[1]{\texttt{\textbackslash#1}}% command name -- adds backslash automatically
\newcommand{\docopt}[1]{\ensuremath{\langle}\textrm{\textit{#1}}\ensuremath{\rangle}}% optional command argument
\newcommand{\docarg}[1]{\textrm{\textit{#1}}}% (required) command argument
\newenvironment{docspec}{\begin{quote}\noindent}{\end{quote}}% command specification environment
\newcommand{\docenv}[1]{\textsf{#1}}% environment name
\newcommand{\docpkg}[1]{\texttt{#1}}% package name
\newcommand{\doccls}[1]{\texttt{#1}}% document class name
\newcommand{\docclsopt}[1]{\texttt{#1}}% document class option name

% Package for title style
\usepackage{sectsty}
\usepackage[utf8]{inputenc}

% Sets section number style
\setcounter{secnumdepth}{3} % uncomment this, if desired
\renewcommand\thesection{\color{white}\arabic{section}}

% Sets title style
\makeatletter
  \renewcommand{\paragraph}{\@startsection{paragraph}%
    {4}{\z@}{-1ex \@plus -1ex \@minus -.3ex}%
    {0.5ex \@plus .2ex}{\normalfont\normalsize\bfseries}}
\makeatother

\makeatletter
  \renewcommand{\subsection}{\@startsection{subsection}%
    {3}{-1.8em}{-3ex \@plus -1ex \@minus -.2ex}%
    {1.5ex \@plus .2ex}
    {\hspace*{-5.5em}\fcolorbox{ltblue}{ltblue}{\parbox[c][1.0ex][b]{4em}{\phantom{space}}}
    \normalfont\large\itshape\color{ltblue}}}
\makeatother

\makeatletter
  \renewcommand{\section}{\@startsection{section}%
    {3}{-1.01em}{-3ex \@plus -1ex \@minus -.2ex}%
    {1.5ex \@plus .2ex}
    {\hspace*{-5.5em}\fcolorbox{blue}{blue}{\parbox[c][1.0ex][b]{4em}{\phantom{space}}}
    \normalfont\Large\itshape\color{blue}}}
\makeatother

% Sets theorem style
\usepackage{thmtools}
\usepackage{transparent}
\definecolor{theb}{rgb}{0.67, 0.80, 0.91}

\declaretheorem[shaded={bgcolor=Lavender,
    textwidth=30em}]{Definition} % Colorbox-styled Theorem 
\declaretheorem[shaded={bgcolor=Lavender,
    textwidth=30em}]{Theorem} % Colorbox-styled Theorem 
\declaretheorem[shaded={bgcolor=Thistle,
    textwidth=30em}]{Corollary} % Colorbox-styled Theorem
    \declaretheorem[shaded={bgcolor=PeachPuff,
        textwidth=30em}]{Proposition} % Colorbox-styled Proposition
\declaretheorem[shaded={rulecolor=Lavender,
    rulewidth=2pt, bgcolor={rgb}{1,1,1}}]{Example-1} % Colorbounded-styled Theorem
\declaretheorem[thmbox=L]{boxtheorem L} % Theorem box L-size
\declaretheorem[thmbox=M]{Example} % Theorem box M-size
\declaretheorem[thmbox=S]{Formula} % Theorem box S-size
\declaretheorem[thmbox=S]{Statement} % Theorem box S-size

% set up pdf bookmark depth
\hypersetup{bookmarksdepth=3}

% ------------------------------------------------------------
\title{MATH240: Discrete Structures}
\author[Matthew He]{Matthew He}
  % if the \date{} command is left out, the current date will be used

% Beginning of the document
\begin{document}

\maketitle% this prints the handout title, author, and date

% abstract
\begin{abstract}
\noindent Abstract
\end{abstract}

% main text
\section{Predicate Logic}
    \begin{Definition}[Predicate]
    A \it predicate \rm is a statement containing some number
    of variable coming from a universe \textbf{u}. aaa
    \end{Definition}

\subsection{Tutorial}
\begin{enumerate}
    \item $ p \Rightarrow p $: tautology, since 
    $ \equiv \neg p \vee p = 1$ 
    \item $(p \Rightarrow q) \Rightarrow p$: contingency,
    true if $ p=q=1 $, false if $ p=0 $. 
    \marginnote{When $ p = 0 $, $ p \Rightarrow q  $is always true, and 
    a $ture \Rightarrow false (p)$ is always false.}
    
    \item $ (\neg (p \wedge \neg q)) \vee p \equiv 1 $: tautology. 
    \item $ (p \Leftrightarrow q) \wedge (p \Leftrightarrow \neg q )$: contradictory. 
    \item to be filled
\end{enumerate}

\begin{Example}
    Prove that a logical formula is satisfiable iff. its 
    negation is falsifiable.
\end{Example}
\textit{Proof.} A logical formula is satisfiable iff there is an assignment of 
all the variables which makes the formula true. By definition of the negation, this assignment makes the negation of our formula false, 
which means the negation is falsifiable. Proof of the converse is analogous.

\paragraph{Prove or disprove:}
\begin{enumerate}
  \item $ \forall n \in \mathbb{N }, \exists m \in \mathbb{N }, n + m = 0 $ \\
  False. Let's proof its negation: $ \exists n \in \mathbb{N }, \forall m \in \mathbb{N },
  n + m \neq 0$, it's true, e.g., we can let $ n = 2 $.
  \item $ \forall n \in \mathbb{N }, \exists m \in \mathbb{Z }, n + m = 0 $\\
  Ture, Let $ n \in \mathbb{N } $, then choose $ m = -n \in \mathbb{Z }  $, and we get $ m+n=0 $. 
  \item $ \forall n \in \mathbb{N }, \exists k \in \mathbb{N }, (k \geq m) \Rightarrow (k \geq 5n) $: \\
  The statment is equivalent to $ (k < m) \vee (k \geq 5n)$. \\
  For any $ n \in \mathbb{N }$, choose $m = 5n $, then for all $ k \in \mathbb{N } $, we have that statement is true.
\end{enumerate}

\subsection{Proofing statement of the form $ p \Rightarrow q $}
    We assume p is true, prove q, which is showing the case that p is true 
    be q is false can't happen.
    \begin{Example}
        \textit{prop.} If n is an odd integer, then $n^{2} $ is an odd integer.
    \end{Example}
    Write the proposition in predicate formula:
    \[u = \mathbb{Z }, \phantom{x} \forall n: ((\exists k : n = 2k+1) \Rightarrow (\exists n^{2} = 2l+1))\]
    \textit{proof.} Let n be ana integer. Assume that n is odd, that is, there exist k such that $ n = 2k+1 $.\\
    Then, $ n^{2} = (2k+1)^{2} = 2(2k^{2} + 2k) + 1 $. Let $ l = 2k^{2} + 2k $,
    then $ n^{2} = 2l + 1 $, thus, it's odd.

\subsection{To disprove a statement: prove its negation is true}



\begin{Example}
    disprove $ \exists x \forall y: x + y \neq 0  $. \\
    prove $ \forall x \exists y : x+ y = 0 \equiv \neg (\exists x \forall y: x + y \neq 0) $
\end{Example}
\textit{proof.} Let $ x \in \mathbb{R } $ be given. Pick $ y = -x  $, then 
$ x + y = 0 $. Q.E.D

\marginnote{Since $ p \wedge \neg p \equiv 0 $}

\subsection{Converse and Contrapositive}

\begin{Definition}\phantom{x }\\
    \begin{enumerate}
        \item the \textbf{converse} of $ p \Rightarrow q  $ is $  q \Rightarrow p  $\\
        NB. $ p \Rightarrow q  \not\equiv  q \Rightarrow p  $
        \item the \textbf{contrapositive} of $ p \Rightarrow q  $ is$\neg q \Rightarrow \neg p  $ \\
        $ p \Rightarrow q  \equiv \neg p \vee q \equiv \neg q \Rightarrow \neg p$
    \end{enumerate}
\end{Definition}

\subsection{Proofs by contradiction}
You assume something is true, and get something nonsense.\\
\[\neg p \Rightarrow 0 \equiv \neg (\neg p) \vee 0 \equiv p\]

\begin{Example}
    \textit{prop.} there is no least positive rational number.\\
    \[u = \mathbb{Q }: \neg (\exists x : x>0 \wedge (\forall y: y > 0 \Rightarrow x \leq y))\]
\end{Example}
\textit{proof.} Suppose, for a contradiction that the proposition is false, that is,
there exist $  x \in \mathbb{ Q } $ such that $ x>0 $ and for all $ y \in \mathbb{Q } $ with $ y>0, x \leq y  $.\\
Let $ y = \frac{x }{2}$, we have $ \frac{x }{2} > 0 $ since $ x>0 $. Then $ x \leq y  $, so $ x \leq \frac{x }{2} $.\\
Divide through by x ( because x > 0) to get $ 1 \leq \frac{1}{2}$.\\
the contradiction completes the proof. Q.E.D

\subsection{Case Analysis}
\begin{Example}
    \textit{prop.} There exists irrational numbers a,b such that a is rational.
\end{Example}

If 

\section{Functions}
\begin{Definition}[Surjective and injective]
    A function is surjective if 
    \[\forall b \in B, \exists a \in A, f(a) = b\]
    A function is injective if 
    \[\forall a_1, a_2 \in A, a_1 \neq a_2 \rightarrow f(a_1) \neq f(a_2)\]
    or $ a_1 = a_2 \rightarrow f(a_1) = f(a_2) $.
\end{Definition}


\section{Graph Theory}
\subsection{Definitions}
A graph is a pair $ G = (V,E) $, where V is a nonempty set and \[E \subseteq \left\{ {u,v}: u,v \in V, u \neq v\right\}.\]

\marginnote{The graph \( G \) is said to be \textit{finite} if both \( V \) are finite sets.
A simple graph is a graph that does not have more than one edge between any two vertices and no edge
starts and ends at the same vertex.}

\textbf{Degrees and k-regularity}
The \textit{neighbors} of a vertex \( v \) are all \( u \in V \) such that \( uv \in E \). The 
\textit{degree} of a vertex \( v \) is the number of neighbors of \( v \), denoted by \( deg(v) \).
A graph is said to be \( k-regular \) for some \( k\in \mathbb{N} \) if every \( v \in V \) has degree \( k \).

The following theorem relates vertex degree to the number of edges.
\begin{Theorem}
    Let G = (V,E) be a finite graph. Then \[\sum_{v \in V} \deg(v) = 2 \left|E\right|.\]
\end{Theorem}

\begin{Corollary}[Handshaking Lemma]
    In every finite simple graph, the number of vertices having odd degree is even.
\end{Corollary}

From Theorem above, we can derive a corollary that counts the number of edges in k-regular graphs.

\begin{Corollary}
    Let G = (V,E) be k-regular. Then \[\left|E\right| = \frac{k\left|V\right|}{2}.\]
\end{Corollary}

\textbf{Walks, paths, and cycles.} A \textit{walk} in a graph \( G = (V,E) \)
is a sequence of vertices \( \sigma = (v_0, v_1, \ldots, v_k) \) such that \( v_i v_{i+1} \in E \) for all \( i \) with \( 0 \leq i < k \). 
The \textit{endpoints} of the walk are \( v_0 \) and \( v_k \), and the \textit{lenght} of the walk \( \sigma \) is \( \left|\sigma\right| \)
The walk is said to be \textit{closed} if \( v_0 = v_k \) and \textit{open} otherwise. 
\clearpage

A walk is a \textit{path} if no vertices are repeated. 
\begin{Theorem}
    Let G = (V,E) be a graph. If \( u \) and \( v \) are 
    vertices such that there exists a walk from \( u \) to \( v \), then there exists a path from \( u \) to \( v \).
\end{Theorem}

A \textit{cycle} is a walk of length at least 3 and no vertices repeated except for \( v_0 = v_k \).

\begin{Proposition}
    Let G = (V,E). If G contains a closed walk of odd length, then G contains a cycle of odd length.
\end{Proposition}

% end of main text

\makeatletter
  \renewcommand{\section}{\@startsection{section}%
    {3}{0.8em}{-3ex \@plus -1ex \@minus -.2ex}%
    {1.5ex \@plus .2ex}
    {\hspace*{-5.5em}\fcolorbox{Periwinkle}{Periwinkle}{\parbox[c][1.0ex][b]{4em}{\phantom{space}}}
    \normalfont\Large\itshape\color{blue}}}
\makeatother

\bibliography{marginnotes}
\bibliographystyle{plainnat}

\end{document}
