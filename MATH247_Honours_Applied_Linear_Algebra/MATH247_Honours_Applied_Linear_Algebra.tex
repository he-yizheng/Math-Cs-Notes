\documentclass{tufte-handout}

%\geometry{showframe}
%\geometry{showframe}% for debugging purposes -- displays the margins

\usepackage{amsmath}
\usepackage{amssymb}
\usepackage{cleveref}
\crefname{Proposition}{Proposition}{Propositions}
\Crefname{Proposition}{Proposition}{Propositions}
\crefname{Theorem}{Theorem}{Theorems}
\Crefname{Theorem}{Theorem}{Theorems}
\crefname{Definition}{Definition}{Definitions}
\Crefname{Definition}{Definition}{Definitions}
\crefname{Corollary}{Corollary}{Corollaries}
\Crefname{Corollary}{Corollary}{Corollaries}
\crefname{Lemma}{Lemma}{Lemmas}
\Crefname{Lemma}{Lemma}{Lemmas}
\crefname{Example}{Example}{Examples}
\Crefname{Example}{Example}{Examples}
\usepackage{amsthm}

% Set up the images/graphics package
\usepackage{graphicx}
\setkeys{Gin}{width=\linewidth,totalheight=\textheight,keepaspectratio}
\graphicspath{{graphics/}}

% The following package makes prettier tables.  We're all about the bling!
\usepackage{booktabs}

% The units package provides nice, non-stacked fractions and better spacing
% for units.
\usepackage{units}

% The fancyvrb package lets us customize the formatting of verbatim
% environments.  We use a slightly smaller font.
\usepackage{fancyvrb}
\fvset{fontsize=\normalsize}

% Small sections of multiple columns
\usepackage{multicol}

% Provides paragraphs of dummy text
\usepackage{lipsum}

% Defines colors
\usepackage{xcolor}
\definecolor{blue}{cmyk}{0.63, 0.37, 0, 0.57}     
\definecolor{ltblue}{RGB}{78,150,179}

% These commands are used to pretty-print LaTeX commands
\newcommand{\doccmd}[1]{\texttt{\textbackslash#1}}% command name -- adds backslash automatically
\newcommand{\docopt}[1]{\ensuremath{\langle}\textrm{\textit{#1}}\ensuremath{\rangle}}% optional command argument
\newcommand{\docarg}[1]{\textrm{\textit{#1}}}% (required) command argument
\newenvironment{docspec}{\begin{quote}\noindent}{\end{quote}}% command specification environment
\newcommand{\docenv}[1]{\textsf{#1}}% environment name
\newcommand{\docpkg}[1]{\texttt{#1}}% package name
\newcommand{\doccls}[1]{\texttt{#1}}% document class name
\newcommand{\docclsopt}[1]{\texttt{#1}}% document class option name

% Package for title style
\usepackage{sectsty}
\usepackage[utf8]{inputenc}

% Sets section number style
\setcounter{secnumdepth}{3} % uncomment this, if desired
\renewcommand\thesection{\color{white}\arabic{section}}

% Sets title style
\makeatletter
  \renewcommand{\paragraph}{\@startsection{paragraph}%
    {4}{\z@}{-1ex \@plus -1ex \@minus -.3ex}%
    {0.5ex \@plus .2ex}{\normalfont\normalsize\bfseries}}
\makeatother

\makeatletter
  \renewcommand{\subsection}{\@startsection{subsection}%
    {3}{-1.8em}{-3ex \@plus -1ex \@minus -.2ex}%
    {1.5ex \@plus .2ex}
    {\hspace*{-5.5em}\fcolorbox{ltblue}{ltblue}{\parbox[c][1.0ex][b]{4em}{\phantom{space}}}
    \normalfont\large\itshape\color{ltblue}}}
\makeatother

\makeatletter
  \renewcommand{\section}{\@startsection{section}%
    {3}{-1.01em}{-3ex \@plus -1ex \@minus -.2ex}%
    {1.5ex \@plus .2ex}
    {\hspace*{-5.5em}\fcolorbox{blue}{blue}{\parbox[c][1.0ex][b]{4em}{\phantom{space}}}
    \normalfont\Large\itshape\color{blue}}}
\makeatother

% Sets theorem style
\usepackage{thmtools}
\usepackage{transparent}
\definecolor{theb}{rgb}{0.67, 0.80, 0.91}

\declaretheorem[shaded={bgcolor=Lavender,
    textwidth=30em}]{Definition} % Colorbox-styled Theorem 
\declaretheorem[shaded={bgcolor=Lavender,
    textwidth=30em}]{Theorem} % Colorbox-styled Theorem 
\declaretheorem[shaded={bgcolor=Thistle,
    textwidth=30em}]{Corollary} % Colorbox-styled Theorem
\declaretheorem[shaded={bgcolor=PeachPuff,
    textwidth=30em}]{Proposition} % Colorbox-styled Proposition
\declaretheorem[shaded={bgcolor=Thistle,
    textwidth=30em}]{Lemma} % Colorbox-styled Lemma
\declaretheorem[shaded={rulecolor=Lavender,
    rulewidth=2pt, bgcolor={rgb}{1,1,1}}]{Example-1} % Colorbounded-styled Theorem
\declaretheorem[thmbox=L]{boxtheorem L} % Theorem box L-size
\declaretheorem[thmbox=M]{Example} % Theorem box M-size
\declaretheorem[thmbox=S]{Formula} % Theorem box S-size
\declaretheorem[thmbox=S]{Statement} % Theorem box S-size

% set up pdf bookmark depth
\hypersetup{bookmarksdepth=3}

% ------------------------------------------------------------
\title{MATH247: Honours Applied Linear Algebra}
\author[Matthew He]{Matthew He}
  % if the \date{} command is left out, the current date will be used
% Beginning of the document
\begin{document}
\maketitle% this prints the handout title, author, and date

% abstract
\begin{abstract}
\noindent Abstract
\end{abstract}

% main text
\section{Preliminaries}
\subsection{Fields}

\begin{Definition}[Field]
    A (nonempty) set with two (inner) operations, addition and multiplication:\\
    \begin{align*}
      \cdot & : K \times K \mapsto K, \cdot(x,y) = x \cdot y\\
      + & : K \times K \mapsto K, +(x,y) = x + y
    \end{align*}
    is called a \textit{field} if the following axioms hold for all $x,y,z \in K$:\\
    \begin{align*}
      &(F1) \quad x+(y+z) = (x+y)+z, \quad x(yz) = (xy)z, \quad
        \text{ (Associativity)}\\
      &(F2) \quad x+y = y+x, \quad xy = yx, \quad
        \text{ (Commutativity)}\\
      &(F3) \quad x+(y+z) = (x+y)+z, \quad x(yz) = (xy)z, \quad
        \text{ (Distributivity)}\\
      &(F4) \quad \exists o \in K \text{ such that } x+o = x,
        \exists e \in K \text{ such that } x \cdot e = x, \quad
        \text{ (Neutral elements)}\\
      &(F5a) \quad \exists a \in K \text{ such that } x+a = o, \quad
        \text{ (Additive inverses)}\\
      &(F5b) \quad \exists b \in K \text{ such that } x \cdot b = e, \quad
        \text{ (Multiplicative inverses)}
    \end{align*}
\end{Definition}

\textbf{Subfields}
Write \( (K,+,\cdot) \) to point out notation for a field.

\begin{Example}[\( \mathbb{F}_2 \) - finite field]
    Let \( K = \left\{0,1\right\} \) be a set with \( 1 \neq 0 \), and define the
    operations \( +, \cdot \) as follows:
    \begin{align*}
      0+x &= x + 0, \quad 0+1 = 1, \quad 1+0 = 1, \quad 1+1 = 0\\
      0 \cdot 0 &= 0, \quad 0 \cdot 1 = 0, \quad 1 \cdot 0 = 0, \quad 1 \cdot 1 = 1
    \end{align*}
\end{Example}

\begin{Theorem}
    Let \( (K,+,\cdot) \) be a field. Then for all \( x,y,z \):
    \begin{align*}
      &(a) \quad x+y = x+z \Rightarrow y = z \quad (Cancellation)\\
      &(b) \quad xy=xz \Rightarrow y = z, \forall x \in K \setminus \left\{0\right\}\\
      &(c) \quad x \cdot o = o\\
      &(d) \quad x \cdot y = o \Rightarrow x = o \lor y = o \quad \text{(Free of zero divisors)}
    \end{align*}
\end{Theorem}

\textit{Proof.}
(a) Let a be the odd


\subsection{The field of complex number}
Complex numbers \( \mathbb{C} \) is born out of necessity to solve 
equations like \[x^2 + t = 0.\]
\begin{Definition}[Complex number]
We set \[\mathbb{C} = \left\{a+bi|a,b\in \mathbb{R}\right\}\]
where "i" is the imaginary unit \[ i^2=-1 \quad (\star).\]
Using ordinary unit for addition and multiplication in \( \mathbb{R} \)
and \( (\star )\) , \( (\mathbb{C}, + ,\cdot) \) becomes a field containing
\( \mathbb{R} \) as a subfield.
\end{Definition}

Given \( z=a+bi \in \mathbb{C}\) we define:
\begin{align*}
  &\mathbb{z} = a-bi \quad \text{conjugate}\\
  & R(z) = a\\
  & Jm() =b\\
\end{align*}

\begin{Theorem}
    Let \( u,v \in \mathbb{C} \)
\end{Theorem}

% end of main text

\makeatletter
  \renewcommand{\section}{\@startsection{section}%
    {3}{0.8em}{-3ex \@plus -1ex \@minus -.2ex}%
    {1.5ex \@plus .2ex}
    {\hspace*{-5.5em}\fcolorbox{Periwinkle}{Periwinkle}{\parbox[c][1.0ex][b]{4em}{\phantom{space}}}
    \normalfont\Large\itshape\color{blue}}}
\makeatother

\bibliography{marginnotes}
\bibliographystyle{plainnat}

\end{document}
