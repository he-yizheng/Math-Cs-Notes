\documentclass{tufte-handout}

%\geometry{showframe}
%\geometry{showframe}% for debugging purposes -- displays the margins

\usepackage{amsmath}
\usepackage{amssymb}
\usepackage{cleveref}
\crefname{Proposition}{Proposition}{Propositions}
\Crefname{Proposition}{Proposition}{Propositions}
\crefname{Theorem}{Theorem}{Theorems}
\Crefname{Theorem}{Theorem}{Theorems}
\crefname{Definition}{Definition}{Definitions}
\Crefname{Definition}{Definition}{Definitions}
\crefname{Corollary}{Corollary}{Corollaries}
\Crefname{Corollary}{Corollary}{Corollaries}
\crefname{Lemma}{Lemma}{Lemmas}
\Crefname{Lemma}{Lemma}{Lemmas}
\crefname{Example}{Example}{Examples}
\Crefname{Example}{Example}{Examples}
\usepackage{amsthm}

% Set up the images/graphics package
\usepackage{graphicx}
\setkeys{Gin}{width=\linewidth,totalheight=\textheight,keepaspectratio}
\graphicspath{{graphics/}}

% The following package makes prettier tables.  We're all about the bling!
\usepackage{booktabs}

% The units package provides nice, non-stacked fractions and better spacing
% for units.
\usepackage{units}

% The fancyvrb package lets us customize the formatting of verbatim
% environments.  We use a slightly smaller font.
\usepackage{fancyvrb}
\fvset{fontsize=\normalsize}

% Small sections of multiple columns
\usepackage{multicol}

% Provides paragraphs of dummy text
\usepackage{lipsum}

% Defines colors
\usepackage{xcolor}
\definecolor{blue}{cmyk}{0.63, 0.37, 0, 0.57}     
\definecolor{ltblue}{RGB}{78,150,179}

% These commands are used to pretty-print LaTeX commands
\newcommand{\doccmd}[1]{\texttt{\textbackslash#1}}% command name -- adds backslash automatically
\newcommand{\docopt}[1]{\ensuremath{\langle}\textrm{\textit{#1}}\ensuremath{\rangle}}% optional command argument
\newcommand{\docarg}[1]{\textrm{\textit{#1}}}% (required) command argument
\newenvironment{docspec}{\begin{quote}\noindent}{\end{quote}}% command specification environment
\newcommand{\docenv}[1]{\textsf{#1}}% environment name
\newcommand{\docpkg}[1]{\texttt{#1}}% package name
\newcommand{\doccls}[1]{\texttt{#1}}% document class name
\newcommand{\docclsopt}[1]{\texttt{#1}}% document class option name

% Package for title style
\usepackage{sectsty}
\usepackage[utf8]{inputenc}

% Sets section number style
\setcounter{secnumdepth}{3} % uncomment this, if desired
\renewcommand\thesection{\color{white}\arabic{section}}

% Sets title style
\makeatletter
  \renewcommand{\paragraph}{\@startsection{paragraph}%
    {4}{\z@}{-1ex \@plus -1ex \@minus -.3ex}%
    {0.5ex \@plus .2ex}{\normalfont\normalsize\bfseries}}
\makeatother

\makeatletter
  \renewcommand{\subsection}{\@startsection{subsection}%
    {3}{-1.8em}{-3ex \@plus -1ex \@minus -.2ex}%
    {1.5ex \@plus .2ex}
    {\hspace*{-5.5em}\fcolorbox{ltblue}{ltblue}{\parbox[c][1.0ex][b]{4em}{\phantom{space}}}
    \normalfont\large\itshape\color{ltblue}}}
\makeatother

\makeatletter
  \renewcommand{\section}{\@startsection{section}%
    {3}{-1.01em}{-3ex \@plus -1ex \@minus -.2ex}%
    {1.5ex \@plus .2ex}
    {\hspace*{-5.5em}\fcolorbox{blue}{blue}{\parbox[c][1.0ex][b]{4em}{\phantom{space}}}
    \normalfont\Large\itshape\color{blue}}}
\makeatother

% Sets theorem style
\usepackage{thmtools}
\usepackage{transparent}
\definecolor{theb}{rgb}{0.67, 0.80, 0.91}

\declaretheorem[shaded={bgcolor=Lavender,
    textwidth=30em}]{Definition} % Colorbox-styled Theorem 
\declaretheorem[shaded={bgcolor=Lavender,
    textwidth=30em}]{Theorem} % Colorbox-styled Theorem 
\declaretheorem[shaded={bgcolor=Thistle,
    textwidth=30em}]{Corollary} % Colorbox-styled Theorem
\declaretheorem[shaded={bgcolor=PeachPuff,
    textwidth=30em}]{Proposition} % Colorbox-styled Proposition
\declaretheorem[shaded={bgcolor=Thistle,
    textwidth=30em}]{Lemma} % Colorbox-styled Lemma
\declaretheorem[shaded={rulecolor=Lavender,
    rulewidth=2pt, bgcolor={rgb}{1,1,1}}]{Example-1} % Colorbounded-styled Theorem
\declaretheorem[thmbox=L]{boxtheorem L} % Theorem box L-size
\declaretheorem[thmbox=M]{Example} % Theorem box M-size
\declaretheorem[thmbox=S]{Formula} % Theorem box S-size
\declaretheorem[thmbox=S]{Statement} % Theorem box S-size

% set up pdf bookmark depth
\hypersetup{bookmarksdepth=3}

% ------------------------------------------------------------
\title{MATH247: Honours Applied Linear Algebra}
\author[Matthew He]{Matthew He}
  % if the \date{} command is left out, the current date will be used
% Beginning of the document
\begin{document}
\maketitle% this prints the handout title, author, and date

% abstract
\begin{abstract}
\noindent Abstract
\end{abstract}

% main text
\section{Preliminaries}
\subsection{Fields}

\begin{Definition}[Field]
    A (nonempty) set with two (inner) operations, addition and multiplication:\\
    \begin{align*}
      \cdot & : K \times K \mapsto K, \cdot(x,y) = x \cdot y\\
      + & : K \times K \mapsto K, +(x,y) = x + y
    \end{align*}
    is called a \textit{field} if the following axioms hold for all $x,y,z \in K$:\\
    \begin{align*}
      &(F1) \quad x+(y+z) = (x+y)+z, \quad x(yz) = (xy)z, \quad
        \text{ (Associativity)}\\
      &(F2) \quad x+y = y+x, \quad xy = yx, \quad
        \text{ (Commutativity)}\\
      &(F3) \quad x+(y+z) = (x+y)+z, \quad x(yz) = (xy)z, \quad
        \text{ (Distributivity)}\\
      &(F4) \quad \exists o \in K \text{ such that } x+o = x,
        \exists e \in K \text{ such that } x \cdot e = x, \quad
        \text{ (Neutral elements)}\\
      &(F5a) \quad \exists a \in K \text{ such that } x+a = o, \quad
        \text{ (Additive inverses)}\\
      &(F5b) \quad \exists b \in K \text{ such that } x \cdot b = e, \quad
        \text{ (Multiplicative inverses)}
    \end{align*}
\end{Definition}

\textbf{Subfields}
Write \( (K,+,\cdot) \) to point out notation for a field.

\begin{Example}[\( \mathbb{F}_2 \) - finite field]
    Let \( K = \left\{0,1\right\} \) be a set with \( 1 \neq 0 \), and define the
    operations \( +, \cdot \) as follows:
    \begin{align*}
      0+x &= x + 0, \quad 0+1 = 1, \quad 1+0 = 1, \quad 1+1 = 0\\
      0 \cdot 0 &= 0, \quad 0 \cdot 1 = 0, \quad 1 \cdot 0 = 0, \quad 1 \cdot 1 = 1
    \end{align*}
\end{Example}

\begin{Theorem}
    Let \( (K,+,\cdot) \) be a field. Then for all \( x,y,z \):
    \begin{align*}
      &(a) \quad x+y = x+z \Rightarrow y = z \quad (Cancellation)\\
      &(b) \quad xy=xz \Rightarrow y = z, \forall x \in K \setminus \left\{0\right\}\\
      &(c) \quad x \cdot o = o\\
      &(d) \quad x \cdot y = o \Rightarrow x = o \lor y = o \quad \text{(Free of zero divisors)}
    \end{align*}
\end{Theorem}

\textit{Proof.}
(a) Let a be the odd


\subsection{The field of complex number}
Complex numbers \( \mathbb{C} \) is born out of necessity to solve 
equations like \[x^2 + t = 0.\]
\begin{Definition}[Complex number]
We set \[\mathbb{C} = \left\{a+bi|a,b\in \mathbb{R}\right\}\]
where "i" is the imaginary unit \[ i^2=-1 \quad (\star).\]
Using ordinary unit for addition and multiplication in \( \mathbb{R} \)
and \( (\star )\) , \( (\mathbb{C}, + ,\cdot) \) becomes a field containing
\( \mathbb{R} \) as a subfield.
\end{Definition}

Given \( z=a+bi \in \mathbb{C}\) we define:
\begin{align*}
  &\mathbb{z} = a-bi \quad \text{conjugate}\\
  & R(z) = a\\
  & Jm() =b\\
\end{align*}

\begin{Theorem}
    Let \( u,v \in \mathbb{C} \)
\end{Theorem}


\subsection{Matrices over a field}
Let \( (K,+,\cdot) \) be a field. Generalizing the concept of a matrix over \( K\in \left\{\mathbb{R,C}\right\} \),
for \( m,n\in \mathbb{N} \), we define:
\[
K^{m \times n} := \left\{ 
\begin{pmatrix}
a_{11} & \cdots & a_{1n} \\
\vdots & \ddots & \vdots \\
a_{m1} & \cdots & a_{mn}
\end{pmatrix} 
\;\middle|\; a_{ij} \in K \; (i = 1, \ldots, m, \; j = 1, \ldots, n)
\right\}
\]
of all $m \times n$ matrices with entries in $K$. \\
Given a matrix $A \in K^{m \times n}$, we write $a_{ij}$ for its entry in the $i$-th row and $j$-th column or $A[i, j]$.

We make the convention 
\[
K^n := K^{n \times 1}.
\]

Inspired by the case $K \in \{\mathbb{R}, \mathbb{C}\}$, we define the following calculus for matrices in $K^{m \times n}$:
\begin{itemize}
    \item \textbf{Addition:} For $A, B \in K^{m \times n}$ we have $(A + B)[i, j] = A[i, j] + B[i, j]$.
    \item \textbf{Multiplication:} For $A \in K^{m \times n}$ and $B \in K^{n \times p}$ we have 
    \[
    (AB)[i, k] = \sum_{j=1}^n A[i, j] B[j, k].
    \]
    \item \textbf{Scalar Multiplication:} For $A \in K^{m \times n}$ and $\lambda \in K$ we have $(\lambda A)[i, j] = \lambda A[i, j]$.
\end{itemize}

The zero matrix in $K^{m \times n}$ is the matrix of all zeros, denoted by $\mathbf{0}_{m \times n}$ or simply $\mathbf{0}$. For $m = n$, we write $\mathbf{0}_n := \mathbf{0}_{n \times n}$.

The matrix units $E_{ij} \in K^{m \times n}$ with
\[
E_{ij} =
\begin{pmatrix}
\cdots & \cdots & \cdots & \cdots \\
\cdots & 1 & \cdots & \cdots \\
\cdots & \cdots & \cdots & \cdots
\end{pmatrix}
\]
(having $1$ at the $(i, j)$-th entry and $0$ elsewhere).

The notion of the transpose of a matrix was also established in MATH 133: For $A \in K^{m \times n}$, its transpose $A^T \in K^{n \times m}$ is defined via 
\[
A^T[i, j] = A[j, i] \quad \text{for all } i = 1, \ldots, n \text{ and } j = 1, \ldots, m.
\]

We call $A \in K^{n \times n}$ a \textbf{square matrix} or simply \textbf{square}. Important classes of square matrices are defined below:
\begin{itemize}
    \item A matrix $A \in K^{n \times n}$ is called an \textbf{upper triangular matrix} if 
    \[
    a_{ij} = 0 \quad \text{for } i > j \quad (i, j = 1, \ldots, n).
    \]
    \item A matrix $A$ is called a \textbf{lower triangular matrix} if 
    \[
    a_{ij} = 0 \quad \text{for } i < j \quad (i, j = 1, \ldots, n).
    \]
\end{itemize}

A matrix is called a \textbf{diagonal matrix} if and only if it is both upper and lower triangular matrix.


A matrix $A = (a_{ij})$ is clearly diagonal if and only if $a_{ij} = 0$ for all $i \neq j$. A special diagonal matrix in $K^{n \times n}$ is the \textbf{identity matrix}
\[
I := I_n :=
\begin{pmatrix}
1 & 0 & \cdots & 0 \\
0 & 1 & \cdots & 0 \\
\vdots & \vdots & \ddots & \vdots \\
0 & 0 & \cdots & 1
\end{pmatrix}.
\]

Using the \textbf{Kronecker delta}
\[
\delta_{ij} =
\begin{cases}
1, & \text{if } i = j, \\
0, & \text{otherwise},
\end{cases}
\]
we can write $I = (\delta_{ij})$.

For a square matrix $A \in K^{n \times n}$, if there exists $B \in K^{n \times n}$ such that
\[
AB = I_n = BA,
\]
we say that $A$ is \textbf{invertible} and write $A^{-1} = B$ and call $A^{-1}$ the \textbf{inverse} of $A$, which is uniquely determined already by one of the conditions $AB = I$ or $BA = I$, as we will confirm in Section 4.1.

The following properties are established in MATH 133 for the case $K \in \{\mathbb{R}, \mathbb{C}\}$ and will be used throughout.

\subsubsection*{Theorem 1.2.1 (Matrix calculus)}
Let $A, A' \in K^{m \times n}$, $B, B' \in K^{n \times r}$, and $C \in K^{r \times s}$. Then the following hold:
\begin{enumerate}
    \item[(a)] $A(B + B') = AB + AB'$ and $(B + B')C = BC + B'C$ \hfill \textit{(distributivity)}.
    \item[(b)] $\lambda(AB) = (\lambda A)B = A(\lambda B)$ for all $\lambda \in K$ \hfill \textit{(homogeneity)}.
    \item[(c)] $(AB)C = A(BC)$ \hfill \textit{(associativity)}.
    \item[(d)] $I_m A = AI_n = A$.
    \item[(e)] $(AB)^T = B^T A^T$.
    \item[(f)] $A, B$ are invertible, then $(AB)^{-1} = B^{-1}A^{-1}$. In particular, $AB$ is invertible.
    \item[(g)] If $A$ is invertible, then $(A^T)^{-1} = (A^{-1})^T$. In particular, $A^T$ is invertible.
\end{enumerate}

\marginnote{This works because the properties of the field \( K \). However,
matricies over a field is not a field.}


\section{Vector spaces}
We start with the abstract definition of a vector space over a field $K$. Although we are mainly interested in the case $K \in \{\mathbb{R}, \mathbb{C}\}$, it is of some benefit to keep things more general where no extra effort is needed.

As advertised, we will now write $0$ and $1$ for the neutral elements of addition and multiplication in the field $K$.

\subsubsection*{Definition 2.1.1 (Vector space)}
A nonempty set $V$ is called a \textbf{vector space} over the field $(K, +, \cdot)$ (with additive/multiplicative neutral $0$ and $1$, respectively) if there exist operations
\[
\odot : K \times V \to V, \quad (\lambda, x) \mapsto \lambda \odot x \quad \text{(scalar multiplication)}
\]
and
\[
\oplus : V \times V \to V, \quad (x, y) \mapsto x \oplus y \quad \text{(vector addition)}
\]

\marginnote{These are all abstract. It says nothings about how the scalar multiplication works and how it looks like.
We are not defining anything here. We are saying if we have these operations satisfying these properties, then we have a vector space.}

such that the following hold:
\begin{itemize}
    \item[(V1)] $(V, \oplus)$ is an \textbf{abelian group}:
    \begin{itemize}
        \item[(i)] $(x \oplus y) \oplus z = x \oplus (y \oplus z)$ for all $x, y, z \in V$ \hfill \textit{(associativity)};
        \item[(ii)] There exists $o \in V$ such that $o \oplus x = x$ for all $x \in V$ \hfill \textit{(neutral element)};
        \item[(iii)] For every $x \in V$ there exists $-x \in V$ such that $x \oplus (-x) = o$ \hfill \textit{(inverse element)};
        \item[(iv)] $x \oplus y = y \oplus x$ for all $x, y \in V$ \hfill \textit{(commutativity)}.
    \end{itemize}
    \item[(V2)] For all $\lambda, \mu \in K$, $x \in V$ we have
    \[
    (\lambda \cdot \mu) \odot x = \lambda \odot (\mu \odot x) \quad \text{\textit{(mixed associativity)}}.
    \]
    \item[(V3)] For all $\lambda, \mu \in K$, $x, y \in V$ we have
    \[
    \lambda \odot (x \oplus y) = (\lambda \odot x) \oplus (\lambda \odot y) \quad \text{and} \quad (\lambda + \mu) \odot x = (\lambda \odot x) \oplus (\mu \odot x) \quad \text{\textit{(mixed distributivity)}}.
    \]
    \item[(V4)] For all $x \in V$ we have
    \[
    1 \odot x = x \quad \text{\textit{(identity law)}}.
    \]
\end{itemize}

The only correct definition of a vector is it's an element of a vector space.

% end of main text

\makeatletter
  \renewcommand{\section}{\@startsection{section}%
    {3}{0.8em}{-3ex \@plus -1ex \@minus -.2ex}%
    {1.5ex \@plus .2ex}
    {\hspace*{-5.5em}\fcolorbox{Periwinkle}{Periwinkle}{\parbox[c][1.0ex][b]{4em}{\phantom{space}}}
    \normalfont\Large\itshape\color{blue}}}
\makeatother

\bibliography{marginnotes}
\bibliographystyle{plainnat}

\end{document}
