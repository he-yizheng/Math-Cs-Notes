\documentclass{tufte-handout}

%\geometry{showframe}
%\geometry{showframe}% for debugging purposes -- displays the margins

\usepackage{amsmath}
\usepackage{amssymb}
\usepackage{cleveref}
\crefname{Proposition}{Proposition}{Propositions}
\Crefname{Proposition}{Proposition}{Propositions}
\crefname{Theorem}{Theorem}{Theorems}
\Crefname{Theorem}{Theorem}{Theorems}
\crefname{Definition}{Definition}{Definitions}
\Crefname{Definition}{Definition}{Definitions}
\crefname{Corollary}{Corollary}{Corollaries}
\Crefname{Corollary}{Corollary}{Corollaries}
\crefname{Lemma}{Lemma}{Lemmas}
\Crefname{Lemma}{Lemma}{Lemmas}
\crefname{Example}{Example}{Examples}
\Crefname{Example}{Example}{Examples}
\usepackage{amsthm}

% Set up the images/graphics package
\usepackage{graphicx}
\setkeys{Gin}{width=\linewidth,totalheight=\textheight,keepaspectratio}
\graphicspath{{graphics/}}

% The following package makes prettier tables.  We're all about the bling!
\usepackage{booktabs}

% The units package provides nice, non-stacked fractions and better spacing
% for units.
\usepackage{units}

% The fancyvrb package lets us customize the formatting of verbatim
% environments.  We use a slightly smaller font.
\usepackage{fancyvrb}
\fvset{fontsize=\normalsize}

% Small sections of multiple columns
\usepackage{multicol}

% Provides paragraphs of dummy text
\usepackage{lipsum}

% Defines colors
\usepackage{xcolor}
\definecolor{blue}{cmyk}{0.63, 0.37, 0, 0.57}     
\definecolor{ltblue}{RGB}{78,150,179}

% These commands are used to pretty-print LaTeX commands
\newcommand{\doccmd}[1]{\texttt{\textbackslash#1}}% command name -- adds backslash automatically
\newcommand{\docopt}[1]{\ensuremath{\langle}\textrm{\textit{#1}}\ensuremath{\rangle}}% optional command argument
\newcommand{\docarg}[1]{\textrm{\textit{#1}}}% (required) command argument
\newenvironment{docspec}{\begin{quote}\noindent}{\end{quote}}% command specification environment
\newcommand{\docenv}[1]{\textsf{#1}}% environment name
\newcommand{\docpkg}[1]{\texttt{#1}}% package name
\newcommand{\doccls}[1]{\texttt{#1}}% document class name
\newcommand{\docclsopt}[1]{\texttt{#1}}% document class option name

% Package for title style
\usepackage{sectsty}
\usepackage[utf8]{inputenc}

% Sets section number style
\setcounter{secnumdepth}{3} % uncomment this, if desired
\renewcommand\thesection{\color{white}\arabic{section}}

% Sets title style
\makeatletter
  \renewcommand{\paragraph}{\@startsection{paragraph}%
    {4}{\z@}{-1ex \@plus -1ex \@minus -.3ex}%
    {0.5ex \@plus .2ex}{\normalfont\normalsize\bfseries}}
\makeatother

\makeatletter
  \renewcommand{\subsection}{\@startsection{subsection}%
    {3}{-1.8em}{-3ex \@plus -1ex \@minus -.2ex}%
    {1.5ex \@plus .2ex}
    {\hspace*{-5.5em}\fcolorbox{ltblue}{ltblue}{\parbox[c][1.0ex][b]{4em}{\phantom{space}}}
    \normalfont\large\itshape\color{ltblue}}}
\makeatother

\makeatletter
  \renewcommand{\section}{\@startsection{section}%
    {3}{-1.01em}{-3ex \@plus -1ex \@minus -.2ex}%
    {1.5ex \@plus .2ex}
    {\hspace*{-5.5em}\fcolorbox{blue}{blue}{\parbox[c][1.0ex][b]{4em}{\phantom{space}}}
    \normalfont\Large\itshape\color{blue}}}
\makeatother

% Sets theorem style
\usepackage{thmtools}
\usepackage{transparent}
\definecolor{theb}{rgb}{0.67, 0.80, 0.91}

\declaretheorem[shaded={bgcolor=Lavender,
    textwidth=30em}]{Definition} % Colorbox-styled Theorem 
\declaretheorem[shaded={bgcolor=Thistle,
    textwidth=30em}]{Theorem} % Colorbox-styled Theorem 
\declaretheorem[shaded={bgcolor=Thistle,
    textwidth=30em}]{Corollary} % Colorbox-styled Theorem
\declaretheorem[shaded={bgcolor=PeachPuff,
    textwidth=30em}]{Proposition} % Colorbox-styled Proposition
\declaretheorem[shaded={bgcolor=Thistle,
    textwidth=30em}]{Lemma} % Colorbox-styled Lemma
\declaretheorem[shaded={rulecolor=Lavender,
    rulewidth=2pt, bgcolor={rgb}{1,1,1}}]{Example-1} % Colorbounded-styled Theorem
\declaretheorem[thmbox=L]{boxtheorem L} % Theorem box L-size
\declaretheorem[thmbox=M]{Example} % Theorem box M-size
\declaretheorem[thmbox=S]{Formula} % Theorem box S-size
\declaretheorem[thmbox=S]{Statement} % Theorem box S-size

% set up pdf bookmark depth
\hypersetup{bookmarksdepth=3}

% ------------------------------------------------------------
\title{MATH547: Stochastic Processes}
\author[Matthew He]{Matthew He}
  % if the \date{} command is left out, the current date will be used
% Beginning of the document
\begin{document}
\maketitle% this prints the handout title, author, and date

% abstract
\begin{abstract}
\noindent Abstract
\end{abstract}

% main text
\section{Preliminaries}

\begin{Definition}[Probability space]\hfill\\
    
\end{Definition}

\subsection{Independence}
Probability theory is about finding some sort of independence.
\begin{Definition}[Independence]\hfill\\
    \textbf{Independence of Events}\\
    Events \( A_1,A_2,A_3,\ldots,A_n \) are \textit{independent} if
    and only if:\\
    For any collection \( I \subseteq \left\{1,2,\ldots,n\right\} \)
    \[P\left(\bigcap_{j\in I}A_j\right) = \prod_{j\in I}P(A_j).\]
    \textbf{Independence of Random Variables}\\
    For random variables \({X}_1, {X}_2, \ldots {X}_n \),
    taking values in state spaces \({S}_1, {S}_2, \ldots {S}_n \),
    these are \textit{independent} if and only if:\\
    for any \( E_1 \subseteq S_1,E_2 \subset S_2, \ldots E_n \subset S_n\)
    the events \( A_j = \left\{X_j \in E_j\right\} \) for \( 1\leq j \leq n \)
    are independent.
\end{Definition}

\subsection{Convergences and limit theorems}
\begin{Definition}[In-probability convergence]\hfill\\
    The sequence \( (X_j:j \in \mathbb{N}_0) \) converges in-probability to \( X_0 \) if for all \( \epsilon > 0 \),
        \[ \lim_{j\rightarrow\infty}P(d(X_j - X_0) > \epsilon) = 0, \]
        in which case we write \(X_j \xrightarrow[\mathrel{j \to \infty}]{\mathrm{Pr}} X_0\).
\end{Definition}


\begin{Theorem}[Law of large number (weak)]
    If \( \left\{X_n\right\}_{n=0}^{\infty} \) are real-valued,
    independent, identically distributed random variables and
    \(\mathbb{E}(X_i) = \mu < \infty\), then
    \[\frac{1}{n} \sum_{j=1}^{n}X_j\rightarrow^{\mathbb{P}}_{n\rightarrow\infty}=\mathbb{E}(Xi)\]
\end{Theorem}

\begin{Definition}[Almost sure convergence]\hfill\\
    The sequence $(X_j : j \in \mathbb{N}_0)$ converges almost surely to $X_0$ if
\[
\Pr\left(\limsup_{j \to \infty} d(X_j, X_0) > 0\right) = 0,
\]
in which case we write
\(
X_j \xrightarrow[\mathrel{j \to \infty}]{\mathrm{a.s.}} X_0.
\)

In-probability convergence does not rule out rare events.
    
\end{Definition}

\marginnote{The intuition behind the limsup here? Limsup always exist.}

It is generally the case that in-probability convergence is strictly weaker than almost sure convergence.

\begin{Theorem}[Law of large number (strong)]
    If \( \left\{X_n\right\}_{n=0}^{\infty} \) are real-valued,
    independent, identically distributed random variables and
    \(\mathbb{E}(X_i) = \mu < \infty\), then
    \[\frac{1}{n} \sum_{j=1}^{n}X_j\rightarrow^{\text{a.s.}}_{n\rightarrow\infty}=\mathbb{E}(Xi)\]
\end{Theorem}

\marginnote{Large number theorem tells us value will converge to the expected value. Then
central limit theorem tells us how close it will be.}

\begin{Example}[In-probability v.s. almost sure convergence]
    Suppose \( \left\{X_j\right\}^{\infty} \) be independent  \( \left\{0,1\right\} \) - valued
    random variables.
    \[P(X_j=1) = 1\slash j\]
\end{Example}

In this case, we have 
\[\lim_{j\rightarrow \infty } P(\left|X_j-0\right| > \epsilon) = P(X_j=1) = 1\slash j = 0,\]
therefore \(X_j \xrightarrow[\mathrel{j \to \infty}]{\mathrm{Pr}} 0\). However, 
\[P(\limsup_{j\rightarrow\infty} \left|X_j-0\right| > 0) = P(\limsup_{j\rightarrow\infty} X_j = 1) = 1,\]

\begin{Definition}[Weak convergence]
    The sequence $(X_j : j \in \mathbb{N}_0)$ converges in law to $X_0$ if for all bounded continuous functions $\varphi : S \to \mathbb{R}$
\[
\lim_{j \to \infty} \mathbb{E} \varphi(X_j) = \mathbb{E} \varphi(X_0),
\]
in which case we write 
\(
X_j \xrightarrow[\mathrel{j \to \infty}]{\mathrm{law}} X_0.
\)
\end{Definition}

\begin{Theorem}[Central limit theorem]
    If \( \left\{X_n\right\}_{n=0}^{\infty} \) are iid. random variables,
    with \(\mathbb{E}(X_i) = \mu < \infty\) and \(\text{Var}(X_i) = \sigma^2 < \infty\),
    then \( \forall t \in \mathbb{R} \),
    \[\lim_{n\rightarrow\infty}P\left(\frac{\sum_{j=1}^{n}X_j - n\mu}{\sigma\sqrt{n}}\leq t\right) = \Phi(t)\]
    where \(\Phi(t)\) is the standard normal distribution function.
\end{Theorem}

\subsection{Conditional probability and expectation}
\begin{Definition}[Conditional expectation on random variables]\hfill\\
    If X takes countably many values and Y is a random variable
    with either \( Y \geq 0 \) or \( \mathbb{E}\left|Y\right| < \infty \),
    then,
    \begin{align*}
    \mathbb{E}\left[Y|X\right]&=\sum_{X}\mathbb{E}\left[Y|X=x\right]\mathbb{1}(X=x)\\
    &= \sum_{X}\frac{\mathbb{E}\left[Y|\mathbb{1}(X=x)\right]}{P(X=x)}\mathbb{1}(X=x).
    \end{align*}

    Or, in measure-theoretic terms, we have:
    \[\mathbb{E}\left[Y|X\right]=\mathbb{E}\left[Y|\sigma(X)\right]\]
    
\end{Definition}

Another case where we can condition:

\begin{Definition}[Conditioning on families of independent random variables]
    If \( (X_1,X_2,\ldots,X_n) \) are independent random variables in some
    state space S, and \( Y=f(X_1,X_2,\ldots,X_n) \) for \( f:S^n \mapsto \mathbb{R} \)
    \[\]
\end{Definition}


\section{Stochastic Processes}
A stochastic process is a family of random variables indexed by natural numbers (time).

\begin{Definition}[Stochastic process]
    A stochastic process is a collection of random variables
    \(\left\{X_t\right\}_{t\in \mathbb{N}_0}\) where \(t\) represents time.
\end{Definition}

\marginnote{Basic examples:\\
    1. iid. randomHello Professor Masset! I was wondering if you would be free to meet up for a quick discussion about the things this term? We could meet whenever works best for you. If you're busy we can also just meet after the lab meeting starts 

    Smile eyes variables (e.g. iid. dice rolls)\\
    2. Random walk} 

\begin{Definition}[Filtration]
    A \textit{filtration} $(\mathcal{F}_j : j \geq j_0)$ is a sequence of $\sigma$-algebras 
with the property that they are increasing, so for all $j \geq j_0$, 
$\mathcal{F}_j \subseteq \mathcal{F}_{j+1}$. A stochastic process 
$(X_j : j \geq j_0)$ is \textit{adapted} to a filtration if $X_j$ is $\mathcal{F}_j$-measurable for all $j \geq j_0$. 
Any stochastic process also gives rise to a filtration, its \textit{natural filtration}, 
just by setting $\mathcal{F}_j = \sigma(X_k : j_0 \leq k \leq j)$.
\end{Definition}

\begin{Definition}[Countable state space markov chain]
    A stochastic process \(\left\{X_j\right\}_{j\in \mathbb{N}_0}\)
    satisfies the Markov property.
\end{Definition}

\( \sigma \)-algebra is about information.

\subsubsection*{Why We Can Split the Event}
We use the law of total probability (or the chain rule for conditional probabilities) to split the event
$\bigcap_{i=1}^{m+1} \{X_{k+i} \in A_{k+i}\}$ at $X_{k+1}$. By conditioning on $X_{k+1}$, we factor
the probability and then apply the induction hypothesis for the remaining terms.

\subsubsection*{Explanation of the Induction}
The key is that for $n=m+1$, we split the event $\bigcap_{i=1}^{m+1}\{X_{k+i} \in A_{k+i}\}$ at $X_{k+1}$, 
apply the induction hypothesis for the remaining $m$ steps, and then use (a) to reduce the conditioning 
to $X_k$ alone. This confirms that each additional step depends only on the immediately preceding state.


% end of main text

\makeatletter
  \renewcommand{\section}{\@startsection{section}%
    {3}{0.8em}{-3ex \@plus -1ex \@minus -.2ex}%
    {1.5ex \@plus .2ex}
    {\hspace*{-5.5em}\fcolorbox{Periwinkle}{Periwinkle}{\parbox[c][1.0ex][b]{4em}{\phantom{space}}}
    \normalfont\Large\itshape\color{blue}}}
\makeatother

\bibliography{marginnotes}
\bibliographystyle{plainnat}

\end{document}
